\chapter*{Suplement: Rejestr wykorzystania narzędzi AI}
\addcontentsline{toc}{chapter}{Suplement: Rejestr wykorzystania narzędzi AI}

W niniejszym suplemencie zamieszczono przykładowe zapytania (prompty) skierowane do narzędzi Generative AI, które wspomogły kluczowe etapy realizacji projektu FasterPost.

\section*{Kategoria: Logika i Algorytmy}

Poniższy prompt posłużył do zaprojektowania wstępnej koncepcji algorytmu optymalizacji tras kurierskich (Last Mile).

\begin{prompt}[htbp]
\begin{lstlisting}[basicstyle=\footnotesize\ttfamily, breaklines=true]
Jako Senior Python Developer, pomóż mi zaprojektować serwis "RouteOptimizer" w Django.

Dane wejściowe:
- Lista punktów (Paczkomaty) z koordynatami (lat, lon).
- Punkt startowy (Magazyn).

Wymagania:
1. Użyj biblioteki geopy do obliczania dystansu (formuła Haversine).
2. Zastosuj algorytm "Nearest Neighbor" (Najbliższy Sąsiad) do wyznaczenia kolejności odwiedzin.
3. Kod ma być czysty, otypowany i uwzględniać obsługę błędów.

Wygeneruj szkielet klasy w Pythonie.
\end{lstlisting}
\caption{Prompt inicjujący prace nad modułem optymalizacji tras}
\label{prmt:tsp}
\end{prompt}

\section*{Kategoria: Frontend i UI}

Prompt wykorzystany do szybkiego prototypowania widoku dla kurierów w technologii Next.js.

\begin{prompt}[htbp]
\begin{lstlisting}[basicstyle=\footnotesize\ttfamily, breaklines=true]
Stwórz komponent React (Next.js App Router) dla "Courier Dashboard".
Stylizacja: Tailwind CSS (Utility-First).

Funkcjonalność:
- Wyświetl listę przesyłek w formie kafelków ("Task Cards").
- Każdy kafelek musi zawierać: Adres, Status przesyłki i przycisk "Zmień status".
- Widok musi być w pełni responsywny (Mobile First), przystosowany do obsługi na smartfonie.
\end{lstlisting}
\caption{Prompt wspierający tworzenie interfejsu aplikacji mobilnej kuriera}
\label{prmt:frontend}
\end{prompt}